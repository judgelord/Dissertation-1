% Options for packages loaded elsewhere
\PassOptionsToPackage{unicode}{hyperref}
\PassOptionsToPackage{hyphens}{url}
%
\documentclass[
  12pt,
  oneside]{memoir}
\usepackage{amsmath,amssymb}
\usepackage{lmodern}
\usepackage{ifxetex,ifluatex}
\ifnum 0\ifxetex 1\fi\ifluatex 1\fi=0 % if pdftex
  \usepackage[T1]{fontenc}
  \usepackage[utf8]{inputenc}
  \usepackage{textcomp} % provide euro and other symbols
\else % if luatex or xetex
  \usepackage{unicode-math}
  \defaultfontfeatures{Scale=MatchLowercase}
  \defaultfontfeatures[\rmfamily]{Ligatures=TeX,Scale=1}
\fi
% Use upquote if available, for straight quotes in verbatim environments
\IfFileExists{upquote.sty}{\usepackage{upquote}}{}
\IfFileExists{microtype.sty}{% use microtype if available
  \usepackage[]{microtype}
  \UseMicrotypeSet[protrusion]{basicmath} % disable protrusion for tt fonts
}{}
\usepackage{xcolor}
\IfFileExists{xurl.sty}{\usepackage{xurl}}{} % add URL line breaks if available
\IfFileExists{bookmark.sty}{\usepackage{bookmark}}{\usepackage{hyperref}}
\hypersetup{
  pdftitle={An Inescapable Reality: Polarization, Prestige, and the US Military in Politics},
  pdfauthor={Peter M. Erickson},
  hidelinks,
  pdfcreator={LaTeX via pandoc}}
\urlstyle{same} % disable monospaced font for URLs
\usepackage[left = 1.25in, right = 1.25in, top = 1.25in, bottom = 1.25in]{geometry}
\usepackage{longtable,booktabs,array}
\usepackage{calc} % for calculating minipage widths
% Correct order of tables after \paragraph or \subparagraph
\usepackage{etoolbox}
\makeatletter
\patchcmd\longtable{\par}{\if@noskipsec\mbox{}\fi\par}{}{}
\makeatother
% Allow footnotes in longtable head/foot
\IfFileExists{footnotehyper.sty}{\usepackage{footnotehyper}}{\usepackage{footnote}}
\makesavenoteenv{longtable}
\usepackage{graphicx}
\makeatletter
\def\maxwidth{\ifdim\Gin@nat@width>\linewidth\linewidth\else\Gin@nat@width\fi}
\def\maxheight{\ifdim\Gin@nat@height>\textheight\textheight\else\Gin@nat@height\fi}
\makeatother
% Scale images if necessary, so that they will not overflow the page
% margins by default, and it is still possible to overwrite the defaults
% using explicit options in \includegraphics[width, height, ...]{}
\setkeys{Gin}{width=\maxwidth,height=\maxheight,keepaspectratio}
% Set default figure placement to htbp
\makeatletter
\def\fps@figure{htbp}
\makeatother
\setlength{\emergencystretch}{3em} % prevent overfull lines
\providecommand{\tightlist}{%
  \setlength{\itemsep}{0pt}\setlength{\parskip}{0pt}}
\setcounter{secnumdepth}{5}

% --- title page design -----------------------

\pagenumbering{gobble}
%\thispagestyle{empty}
\usepackage{titling}
\pretitle{\centering \LARGE \sffamily \bfseries}
\posttitle{\rmfamily \par}
\preauthor{\vskip 24pt 
  \begin{center} 
  \normalsize{By} \\[6pt]}
\postauthor{
  \vskip 48pt 
  \normalsize
  A dissertation submitted in partial fulfillment of the requirements for the degree of \\[3pt]
  \large{\textsc{Doctor of Philosophy\\ (Political Science)}} \\[3pt]
  \normalsize{at the}  \\[3pt]
  \large{\textsc{University of Wisconsin--Madison}}, \\[3pt]
  \large{2021}
  \par
  \end{center}
  \vskip 12pt
}

\predate{
  \vfill

 \noindent Date of final oral examination: \nolinebreak \date{2021-07-30}

}

\postdate{
   \\[6pt] 

 \noindent The dissertation is approved by the following members of the Final Oral Committee: \\

  %\rule{0.5\textwidth}{.4pt} \\
   Eleanor Neff Powell, Booth Fowler Associate Professor, Political Science  \\[0pt]

  %\rule{0.5\textwidth}{.4pt} \\
   Miriam Seifter, Associate Professor, Law  \\[0pt]
  
  %  \rule{0.5\textwidth}{.4pt} \\
   Alexander M.\ Tahk, Associate Professor, Political Science  \\[0pt]
  
 %   \rule{0.5\textwidth}{.4pt} \\
   Susan Webb Yackee, Collins-Bascom Professor, Public Affairs and Political Science  \\[0pt]

  
 % \rule{0.5\textwidth}{.4pt} \\
   David L. Weimer, Edwin E. Witte Professor,  Political Economy  \\[0pt]

  \cleardoublepage
%  \setcounter{page}{1}
%  \pagenumbering{gobble}
%  \pagestyle{simple} 
}

% document design

% --- serif chapter styles ---
\chapterstyle{dash} 
  \renewcommand*{\chaptitlefont}{\Large \centering \sffamily \bfseries} 
  \renewcommand*{\chapnumfont}{\chaptitlefont}
% \chapterstyle{dowding}

% --- page style ---

% upper right numbering
\pagestyle{simple} 

% chapter page style: upper right
\copypagestyle{chapter}{simple}
\makeoddhead{chapter}{}{}{\thepage}
\makeevenhead{chapter}{}{}{\thepage}
\makeoddfoot{chapter}{}{}{}
\makeevenfoot{chapter}{}{}{}

\copypagestyle{part}{simple}
\makeoddhead{part}{}{}{\thepage}
\makeevenhead{part}{}{}{\thepage}
\makeoddfoot{part}{}{}{}
\makeevenfoot{part}{}{}{}


% --- section style -----------------------

% we don't go deeper than subsection (hopefully..?)
\setsecheadstyle{\large \raggedright \bfseries}
\setsubsecheadstyle{\raggedright \bfseries}
\setsubsubsecheadstyle{\raggedright \itshape} 

% numbering depth
\setsecnumdepth{subsubsection}  

 \usepackage{booktabs}
 \usepackage{cleveref}

\renewcommand{\eqref}{\Cref}
\Crefformat{equation}{#2#1#3}

% \usepackage{float}
% \let\origfigure\figure
% \let\endorigfigure\endfigure
% \renewenvironment{figure}[1][2] {
%     \expandafter\origfigure\expandafter[H]
% } {
%     \endorigfigure
% }



% LINKS 
\usepackage{hyperref}
\hypersetup{breaklinks = true,
            bookmarks = true,  
            colorlinks = true,
            citecolor = black,
            urlcolor = black,
            linkcolor = magenta}

% FIG CAPTIONS 
\let\newfloat\relax 
\usepackage{floatrow}
\floatsetup[figure]{capposition=top}
\floatsetup[table]{capposition=top}
\usepackage{setspace}

% inherit spacing names from memoir
%\newcommand{\OnehalfSpacing}{\onehalfspacing}
%\newcommand{\DoubleSpacing}{\doublespacing}
\newcommand{\DefaultSpacing}{\DoubleSpacing}

\setlength{\parindent}{2ex}

% SPACING FOR BLOCK QUOTES 
\expandafter\def\expandafter\quote\expandafter{\quote\OnehalfSpacing}

% FOR KABLE EXTRA 
\usepackage{booktabs}
\usepackage{longtable}
\usepackage{array}
\usepackage{multirow}
\usepackage{wrapfig}
%\usepackage{float}
\usepackage{colortbl}
\usepackage{pdflscape}
\usepackage{tabu}
\usepackage{threeparttable}
\usepackage{threeparttablex}
\usepackage[normalem]{ulem}
\usepackage{makecell}
\usepackage{xcolor}
\usepackage{ragged2e}
\ifluatex
  \usepackage{selnolig}  % disable illegal ligatures
\fi
\usepackage[style=authoryear,]{biblatex}
\addbibresource{dissertation.bib}

\title{An Inescapable Reality: Polarization, Prestige, and the US Military in Politics}
\author{Peter M. Erickson}
\date{March 18, 2022}

\begin{document}
\maketitle
\begin{abstract}
This dissertation examines how varying levels of polarization and military
prestige impact the US military's political involvement. It first argues that the
political behavior of military and civilian actors with respect to the military
should be viewed through the lens of three central principles of civil-military
and the non-interference of the military into certain realms of the state. The dissertation
then argues that the level of polarization and military prestige impact the willingness
of military and civilian actors to adhere to these principles. Using a mixed methods
approach involving both quantitative and historical analysis, this dissertation
provides evidence that patterns of political activity involving the military change
over time. Ultimately, this dissertation contributes to the scholarly understanding
of how domestic factors shape the political activities involving the military in
mature democracies.
\end{abstract}

\cleardoublepage
\setcounter{page}{1}
\pagenumbering{roman}
\tableofcontents

\clearpage

\listoftables

\clearpage

\listoffigures

\RaggedRight

\mainmatter
\pagenumbering{arabic}
\DoubleSpacing
\setlength{\parindent}{2ex}
\newpage

\newpage

\hypertarget{chapter-one}{%
\chapter*{Chapter One}\label{chapter-one}}
\addcontentsline{toc}{chapter}{Chapter One}

\hypertarget{introduction-and-research-question}{%
\chapter{Introduction and Research Question}\label{introduction-and-research-question}}

Scholars, civilians, diplomats, and military leaders regularly discuss why and how militaries intervene in politics. These discussions, moreover, often include the theoretical question of whether military actors \emph{should} intervene in politics, to what degree, and for what purpose.
Though it would be impossible to list all of the contemporary instances in which the military has become entangled in politics, several stand out. From former President Trump's appointment of several retired military officers to key postings within his administration \autocite{brooks_dont_2016}, the removal of Navy Captain Brent Crozier for leaking a letter to the press in which he castigated the chain of command's response to the onset of the COVID-19 pandemic \autocite{associated_press_teddy_2020}, and the remarkable panoply of senior retired military officers excoriating former President Trump's threat to use active duty forces to quiet protests and riots in cities across America during the summer of 2020 \autocite{brooks_let_2020-1}, there is no shortage of events which illustrate the many challenges of dealing with the military during heightened political polarization. What scholars have not fully theorized - nor tested - is why events such as these occur in the first place.

In this dissertation, I theorize about the causes of political behavior involving the military, and measure these behaviors. The central argument emphasized throughout this dissertation is that the patterns of political behavior and the actors who engage in them are largely influenced by key features of the domestic environment in which these actors operate. In particular, I argue that the degree of \emph{political polarization} prevalent in society, and the degree to which the military is \emph{prestigious}, shape the ways in which military and civilian actors behave politically.

In making this argument, I build upon a general ``motives'' and ``opportunities'' framework first advanced by Finer (1962) and later by Taylor (2003) to describe military intervention in politics \autocite{finer_man_1962,taylor_politics_2003}. I claim that the relative levels of polarization and prestige serve similar functions in that each influences the degree to which both sets of actors adhere to what I define as the the central principles of civil-military relations, which are derived and explained later in this chapter. By conducting a systematic exploration of the impacts of polarization and military prestige on civil-military relations, across time and within several domestic contexts, this dissertation sheds light on what is, I argue, a recognized yet under-specified reality: civilian and military leaders each face unique challenges with respect to the conduct of civil-military relations as the levels of polarization and military prestige change.

Examining the impacts of polarization and military prestige on political behaviors that involve the military is an important undertaking from both normative and empirical perspectives. From a normative perspective, this study helps those who study and practice civil-military relations better understand why the interplay between civilian and military leaders varies over time. While scholars often (rightfully) examine the particular leaders who are involved in crises such as wars, as well as the contexts in which crises such as wars take place, this dissertation contends that we need to examine how and why broader political factors fundamentally impact the conduct of civil-military relations.\footnote{For just two scholarly examples of civil-military relations narratives that are set within the context of war, see \textcite{mcmaster_dereliction_1997} on the Vietnam War, and \textcite{kaiser_no_2014} on World War Two.}

If, for a moment, the conduct of civil-military relations can be viewed analogously to a sporting event, then the players who take the field are the civilian and military actors who engage in the conduct of civil-military relations. A significant part of this dissertation examines the stadium and the playing field - \emph{the context} - in which the players perform. Just as different weather conditions (for example, snow or rain during a football game) influence how athletes perform, so too do changing domestic conditions influence how civilian and military actors behave.

Yet there is much at stake empirically as well. It is one thing to suspect that high polarization and high military prestige make the conduct of civil-military relations more challenging, but another to demonstrate how much more challenging, and in what ways these challenges emerge.

This introductory chapter proceeds in three main parts. First, it introduces the general field of civil-military relations. In doing so, this chapter identifies and exposits what I argue are three central principles of American civil-military relations that have been in place throughout the post-World War Two era. These principles are that of civilian control of the military, the non-partisanship of the military institution, and what I define as the principle of non-interference of the military into certain realms of state policy making. The second part of this introductory chapter describes how this dissertation fits within the scholarly literature and the gaps it intends to fill. The third and final part of this chapter then describes the general plan of the overall dissertation and its methodological approach.

\hypertarget{the-field-of-civil-military-relations}{%
\chapter{The Field of Civil-Military Relations}\label{the-field-of-civil-military-relations}}

Truly one of the interdisciplinary topics of scholarly research, the study of civil-military relations spans many academic sub-fields, including political science, history, sociology, and ethics. The study of civil-military relations also covers an incredibly rich and immense set of questions, including the relationship between militaries and governments; the interaction between militaries and societies; the ingredients of strategic and military effectiveness; and the occurrence of coups, to name just a few. Within the field of political science in particular, one of the areas of civil-military relations that is most studied is the relationship between prominent state officials and military leaders. Indeed, the relationship between these actors is what one scholar has termed, ``leadership at the state's apex'' \autocite[380]{brooks_risa_integrating_2019}, as it is here that numerous and weighty decisions that impact a state's security are made, including whether to go to war, how to form and implement effective strategy, how to prepare for the next conflict, and how large the military and its budget should be.

In the most basic sense, the study of civil-military relations wrestles with a question that Peter Feaver has called a ``paradox'': how a state ensures that its military is strong enough to defeat threats without posing a threat to the state itself \autocite[150]{feaver_civil-military_1996}. The challenge of implementing civil-military relations is simple yet profound, for as Feaver notes, ``just as the military must protect the polity from enemies, so must it conduct its own affairs so as to not destroy or prey on the society it is intended to protect'' \autocite[214]{feaver_civil-military_1999}. As a result of these twin facts - that a state needs a military to defeat external threats, and that a military must possess some degree of coercive force in order to do so - the practice of civil-military relations is always filled with some degree of tension.

A number of scholars have offered various solutions for reducing this inherent tension, but none has been more lasting or consequential than the late Samuel Huntington, whose 1957 work, \emph{The Soldier and the State: The Theory and Politics of Civil-Military Relations}, often serves as a starting point in the study of civil-military relations. Although Huntington's theory has endured important and substantive critiques over the years \autocites[for example, see][]{janowitz_professional_1960,finer_man_1962,cohen_supreme_2003,feaver_armed_2003,brooks_paradoxes_2020}, it is essential to briefly exposit Huntington's ideas for achieving effective civil-military relations.

In the section to follow, I briefly summarize Huntington's argument and logic, as well as several prominent critiques of his work. The purpose of summarizing both Huntington \emph{and} his critics is to argue that in spite of the important differences between them, Huntington and his critics share much common ground regarding the aspirational prominence of three central principles, or standards, that help foster healthy and effective civil-military relations. Furthermore, these principles hold normative implications for how civilian and military actors conduct themselves in the political arena.

\hypertarget{huntington-and-the-soldier-and-the-state}{%
\section{Huntington and The Soldier and the State}\label{huntington-and-the-soldier-and-the-state}}

Though it is not always framed as such, \emph{The Soldier and the State} is a book that lies at the very intersection of domestic politics and international relations. Indeed, the fundamental problem that Huntington sets out to solve is how liberal democracies such as the United States can balance the simultaneous demands of maintaining military security and its domestic liberal political character. Writing in the wake of the Korean War, when the danger posed to the United States by the Cold War was quite palpable, Huntington argued that American security requirements constituted a ``functional imperative'' in that they required the US to maintain a military that is sufficiently strong and capable of defeating external threats \autocite[1-3]{huntington_soldier_1957}.

At the same time, Huntington argued that the liberal character of American society generated a ``societal imperative'' such that the pursuit of American security requirements could never cause the US to deviate from its liberal political ideology in fundamental ways, including abandonment of the principle of civilian control of the military \autocite[1-3]{huntington_soldier_1957}. In this way, Huntington's puzzle is essentially an earlier framing of the ``problematique'' pointed out by Feaver decades later: Huntington aims to solve the tension of rendering the military strong enough to protect the United States from the enormous threat posed by the Cold War, yet simultaneously subservient and obedient to civilian power \autocite{feaver_civil-military_1996}.

Huntington forcefully argued that achieving the simultaneous demands of maintaining the state's security and it's liberal political character was made more difficult by the fact that the American military officer corps was different from the rest of American society in several important ways -- a claim that is not without controversy. In particular, Huntington asserted that the military officer corps held to a distinct ``mind'' and ``ethic'' that viewed conflict as part of a ``universal pattern throughout nature,'' and thus, that ``the military view of man is\ldots decidedly pessimistic'' \autocite[62-63]{huntington_soldier_1957}. This mindset, argued Huntington, clashed with the generally liberal views cherished by the majority in American society, views which emphasized the ``natural relation'' of mankind as one of peace rather than war \autocite[90]{huntington_soldier_1957}. In short, Huntington acknowledged that a gulf existed between America's liberal political character and the general ethic held by its military officer corps, a gulf that Huntington believed accounted for the strong suspicion and skepticism Americans have held for standing military forces since its founding \autocite[143-151]{huntington_soldier_1957}.

Huntington argued that these two requirements - securing the United States during the Cold War and maintaining America's liberal political character - posed a fundamentally unique problem for the conduct of American civil-military relations. From the perspective of national security, sizable standing military forces were a requirement that was likely to endure. At the same time, the United States public would never agree to discard or abandon its liberal political essence.

Huntington's solution for solving these twin dilemmas involves the adoption of a scheme he called ``objective civilian control'' of the military \autocite[83]{huntington_soldier_1957}. In essence, objective civilian control required civilian political leaders to grant the military a high degree of professional autonomy, which Huntington claimed would simultaneously facilitate the military's development of its professional expertise and skills and its eschewal of partisan politics. This claim, it should be noted, is not without controversy, and is discussed later in this chapter.

In practice, instituting ``objective civilian control'' of the military requires recognition by both military and civilian actors that each operates within a distinct sphere of figurative territory \autocite[83-85]{huntington_soldier_1957}. Civilian leaders are not to encroach into areas that are the military's purview, which would damage the military's professional autonomy, and conversely, military leaders should not enter the world of partisan politics, as doing so subverts civilian control and the apolitical nature of the military.

A key aspect of Huntingtonian logic that undergirds the concept of objective civilian control centers on the US military officer corps as a professional body. According to Huntington, the nation's corps of military officers constitute a profession in that they, like the members of other professions such as law, medicine, and the clergy, possess several unique attributes, including that of ``expertise, responsibility, and corporateness'' \autocite[8-10]{huntington_soldier_1957}. By expertise, Huntington had in mind the idea that military professionals possess unique ``knowledge and skills'' in the arena of warfare \autocite[8-10]{huntington_soldier_1957} and, using Harold Lasswell's terms, are experts in the ``management of violence'' \autocite[Lasswell quoted in][11]{huntington_soldier_1957}. Regarding the attribute of responsibility, Huntington argued that the military officer corps is to employ such expertise only in the service and ``at the direction'' of the state \autocite[370]{nielsen_american_2012}. And with respect to the attribute of corporateness, Huntington viewed the military officer corps as a group that certifies its own leaders and polices itself, or in the words of Nielsen, ``as a distinct, bureaucratized body, with a common identity fostered through shared educational, training, and service experiences'' \autocite[370]{nielsen_american_2012}.

Among the many reasons why the concept of a profession matters is the fundamental idea that professions must be nurtured and sustained. They do not automatically flourish. In other words, insofar as the military is in fact a profession rather than merely another type of job, the military as a profession is better suited to develop and sustain itself when it has the latitude to exist as such.\footnote{Many scholars of civil-military relations still emphasize this point, particularly at times when the military downsizes or faces significant economic constraints. For one scholar's warning regarding this dynamic, see \textcite{snider_once_2012}.} The concept of objective civilian control, argued Huntington, would provide such an environment.

Importantly, Huntington also argued that civilians would welcome the concept of objective civilian control, as it would render a military that was exceedingly skilled and capable yet also one that obeyed orders and stayed out of politics.\footnote{Huntington asserts that the development of the US military as a profession largely occurred after the US Civil War, and was largely the function of the geographical and social isolation experienced by the military officer corps. Several military historians reject this claim, however, and instead argue that the US military exhibited real signs of professionalization well before the post-US Civil War Era. For instance, Skelton (1992) argues that the US military made substantive strides towards professionalization beginning after the War of 1812 \autocites{skelton_american_1992}[see also][]{heiss_professionalization_2012}. Grandstaff (1998) as well as Connelly (2005) take a more nuanced view, arguing that the process of military professionalization occurred in two distinct waves during the 19th Century, one before and one after the Civil War \autocite{grandstaff_preserving_1998,connelly_american_2005}. In chapter five of this dissertation, which involves the post-US Civil War military and its leaders, I substantiate that even in the process of professionalizing, military leaders and officers in particular both understood and assented to the importance of basic principles of civil-military relations, including that of civilian control and an avoidance of engaging in overtly partisan politics. For one military historian's analysis of Huntington, see \textcite{coffman_long_1991}.} Over the long term, a posture rooted in objective control of the military, so claimed Huntington, would lead to military excellence without tainting the liberal political character of the country, all while maintaining the principle of civilian control of the military \autocite[83-85]{huntington_soldier_1957}.

\hypertarget{critiques-of-huntington}{%
\section{Critiques of Huntington}\label{critiques-of-huntington}}

Critics have assailed Huntington's concept of \emph{objective civilian control} for years, and it would be impossible to list all of those critiques here. But it is important to summarize these critiques, and to draw out their key themes.

The main thrust of critiques of Huntington's argument centers around the stark separation between military and political spheres that Huntington favored. In levying this broad critique, Huntington's critics often point to and assert Clausewitz's famous dictum that war can be thought of as the ``mere continuation of policy by other means'' \autocite[69, 605]{clausewitz_war_1976}. Because the military is an inherently political institution that cannot exist and operate absent from political influence, so contend the critics of Huntington, the concept of objective civilian control, which relies on a rather significant degree of separation between civilian and military spheres, is theoretically flawed. In other words, the inherent interconnectedness between politics and war, these critics of Huntington claim, renders the idea of a military exercising professional autonomy free from politicians both unwise and impractical.

Critics of Huntingtonian theory argue that in democracies, civilian leaders are legally and authoritatively superior to military leaders. Thus, civilians - not military leaders - hold ultimate responsibility for what militaries do and fail to do. Eliot Cohen forcefully illustrates, with examples ranging from Lincoln to Churchill, that the world's greatest heads of state have never abdicated responsibility during wartime, and at exceptional moments, have even reached far into the details of operations to ensure that militaries understood and implemented their directives \autocite{cohen_supreme_2003}. Peter Feaver has parsimoniously captured the essence of this critique of Huntington by noting that the principle of civilian supremacy over the military means that ``civilians have the right to be wrong'' \autocites[117]{feaver_right_2011}{feaver_armed_2003}.

Other critiques of Huntington focus on the harmful effects of Huntington's theory. For instance, Brooks (2020) warns that the pursuit of Huntingtonian theory leads some military officers to develop ``blind spots'' such that these officers engage in detrimental political actions by rationalizing that because they are ``professional'' officers, their actions are apolitical by default \autocite[17]{brooks_paradoxes_2020}. Others warn that facilitation of a strict separation between military and civilian spheres in practice, and especially during wartime, fails to recognize the degree to which military and civilian spheres must overlap in order to develop, implement, and achieve the goals of national security policy \autocite{rapp_civil-military_2015}.

In spite of these insightful and important concerns and critiques of Huntington, Huntington retains an outsized influence in the study and practice of civil-military relations, even today. As Nielsen has argued, any reputable course taught on civil-military relations likely includes Huntington's \emph{The Soldier and the State}, not only because of the ``boldness and ambition'' of Huntington's work, but also because Huntington was truly the first serious scholar who sought to develop a comprehensive theory to help steer the conduct of civil-military relations \autocite[369]{nielsen_american_2012}. Moreover, many of the issues and discussions within US civil-military relations today - how to maintain civilian control of the military, how separate should the military be from society, how much overlap between civilian and military spheres should there be - are all topics that Huntington, albeit perhaps with some error, addressed in some way.

Yet in my view, there is also much that Huntington and his critics agree upon. In particular, I contend that Huntington and his critics agree on the centrality of three central principles that should govern civil-military relations, even if these critics disagree with Huntington with respect to the methods that these principles are best achieved and maintained. These three central principles are: the principle of civilian control; the principle of non-partisanship; and the principle of ``non-interference'' of the military into certain areas or realms of state policy making.

The most obvious principle upon which Huntington and his critics both agree is that of civilian control of the military. To see this, first consider that one of the two chief purposes of Huntington's formulation of the concept of \emph{objective control} (the other being the ability of the military to defeat external threats) in the first place is to ensure civilian control of the military. At the same time, critics of Huntington assail his argument precisely on the grounds that the concept he offers fails to ensure civilian control of the military. For example, Eliot Cohen's descriptive theory of civil-military relations, which he terms ``the unequal dialogue'' \autocite[208-224]{cohen_supreme_2003}, and Feaver's contention that ``civilians have a right to be wrong'' \autocites[117]{feaver_right_2011}{feaver_armed_2003}, both clearly identify the principle of civilian control as the most essential characteristic of healthy civil-military relations.\footnote{Cohen argues strongly that the most successful wartime civil-military relations have been those that have involved ``an unequal dialogue - a dialogue, in that both sides expressed their views bluntly, indeed, sometimes offensively, and not once but repeatedly - and unequal, in that the final authority of the civilian leader was unambiguous and unquestioned\ldots{}''\autocite[209]{cohen_supreme_2003}. Cohen, as is the case with other scholars of civil-military relations who have critiqued Huntington, is concerned that Huntington's concept of objective control prevents civilian leaders from exercising appropriate oversight of the military.}

A second principle that both Huntington and his critics affirm is the non-partisan character of the military. Huntington's critics explicitly acknowledge this. Nielsen, for instance, points out that ``Huntington's principle of objective control has both merits and shortcomings. On the positive side, it preserves democratic control, speaks to the importance of an apolitical military and protects military professionalism'' \autocite[375]{nielsen_american_2012}. And Cohen cautions against an outright dismissal of Huntingtonian thought, warning that ``to reject Huntington's ideas of sequestering issues of policy from those of military administration or operations is to open the way to a military that is politicized and, by virtue of its size and discipline, a potentially dominant actor in the conduct of foreign and internal affairs'' \autocite[264]{cohen_supreme_2003}. And in article appropriately titled, ``Military Officers: Political without Partisanship,'' Mackubin Thomas Owens echoes a similar point made by both Brooks (2020) and Rapp (2015), namely, that military officers need to engage with and understand the political process, but ``without becoming swept up in partisan politics'' \autocite[97]{owens_military_2015}. To be sure, Huntington and his critics disagree on exactly how stark the separation between figurative military and political spheres should be, yet both share a concern that the military avoid inappropriate partisan political entanglement.

Finally, there is a third principle that both Huntington and his critics affirm. However, this principle - that of the \emph{non-interference} of the military - is one that is implied rather than explicitly stated. The principle of \emph{non-interference} of the military centers on the notion that the military should not interfere or seek to influence all realms of state policy making. Even though Huntington's critics take issue with the rather stark separation of civil and military spheres Huntington advocated for, I would argue that both Huntington and his critics agree that the military should not take on the role of the statesman or stateswoman. Directionality and authority are concepts that lay at the heart of the principle of non-interference. Civilian leaders may, at times, become involved in the details of military operations, and it is their right and purview to do so. Huntington and his critics agree, however, that the same is not true of military leaders: military actors may not, of their own accord, involve themselves in issues or topic areas that are not part of the military domain. Huntington and his critics alike would be against the idea of a military dictatorship, for example, in which the military runs all of government. Cohen himself warns that dismissing Huntington's theory, especially ``in states with less-established democratic traditions \emph{(than the United States)}\ldots would open the path to direct military intervention in politics'' \autocite[264, emphasis mine]{cohen_supreme_2003}.

In summary, there is much that Samuel Huntington and his critics disagree about concerning civil-military relations. Huntington advocated for a practical separation of political and military spheres to the fullest degree possible, on the basis that the military - as a profession - will be interested in furthering its own autonomy. His critics, on the other hand, point out that there are limits to how separate military affairs and politics can reside, and that therefore, the concept of a military as a profession can never exist entirely absent from the political context. In spite of these critiques, Huntington and his critics share a concern for at least three overarching principles of civil-military relations. The first is a principle of civilian control of the military. The second is a principle that the military, even if it is a political actor, should avoid partisan influence and entanglement as much as possible. The third principle speaks to the idea of the military not interfering in realms of state policy making that are entirely unrelated or indirectly related to the conduct of military affairs.

\hypertarget{three-central-principles-of-civil-military-relations}{%
\chapter{Three Central Principles of Civil-Military Relations}\label{three-central-principles-of-civil-military-relations}}

In this section, I define and exposit in greater detail each of these three central principles of civil-military relations. I argue that these three principles - civilian control of the military, the non-partisanship of the military institution, and the ``non-interference'' of the military into certain areas of state policy making - constitute a \emph{baseline} set of principles that inform, guide, and facilitate the conduct of healthy and harmonious civil-military relations. Furthermore, both actor types, military and civilian, have a role in adhering to these principles. When when one or more of these central principles is disregarded or violated by either set of actors, a state's civil-military relationship is likely to be discordant in some way.

Though it is not a central aspect of the argument that I make in this dissertation, these ``principles of civil-military relations'' are the exception - rather than the norm - in US history. They emerge from the theoretical and practical expectations of a professional military operating in the post-World War Two era (hence their derivation from Huntington, his critics, and military officers who lived in the mid-late 20th Century).

\hypertarget{central-principle-1-civilian-control-of-the-military}{%
\section{Central Principle 1: Civilian Control of the Military}\label{central-principle-1-civilian-control-of-the-military}}

The first central principle of civil-military relationships is that of civilian control of the military. It is listed first deliberately, as it is the central component by which the health of civil-military relationships, particularly in democracies, is maintained. Without strong adherence to the principle of civilian control of the military, it is doubtful that a democratic state's civil-military relationship can be healthy nor successful in the long run. So, what is civilian control of the military, and how does one know if military and civilian actors within a state are adhering to it as a principle of civil-military relations?

On the surface, the principle of civilian control is fairly straightforward: civilians should be in charge of the military. Yet a significant scholarship has pointed out that focusing on coups or other extreme forms of insubordinate military behavior fails to recognize the myriad other ways in which militaries often challenge the principle of civilian control \autocites{croissant_beyond_2010}[see also][242]{cohen_supreme_2003}{beliakova_erosion_2021}. Finer, for instance, warns that militaries can and will violate the principle of civilian control through ``acts of commission, but also by acts of omission'' \autocite[20]{finer_man_1962}. For this reason, Feaver argues that observers interested in the health of the principle of civilian control should examine the ``patterns'' of civilian control, rather than merely looking for whether the principle exists within a state \autocite[167]{feaver_civil-military_1996}.

Thus, while civilian control may at first glance seem as though it is a straightforward concept, it is actually quite subtle. Scholars share the idea that civilian control more accurately refers to the ``relative political power'' that exists between a nation's armed forces and its civilian leaders \autocites[7]{bruneau_civil-military_2019}[see also][]{brooks_shaping_2008}. Brooks, Golby, and Urben argue more precisely that civilian control refers to ``the extent to which political leaders can realize the goals the American people elected them to accomplish'' \autocite[65]{brooks_crisis_2021}.

Thus, in my view, the degree to which the military adheres to the principle of civilian control involves both outcomes (does the military do what it is told) \emph{and} process (what is the spirit of the military in doing what it is told). This focus on process is important. Military leaders may ultimately obey the orders of their bosses, but along the way engage in a range of behaviors that thwart, stymie, and/or frustrate the will of elected civilian leaders.

The behaviors or actions that challenge the principle of civilian control range from the very subtle to the very obvious. For example, as Brooks, Golby, and Urben argue, military officers may choose to share little information with civilians about an issue, or comply with a civilian directive at a leisurely pace rather than with spirited initiative \autocite{brooks_crisis_2021}. In these types of cases, it is possible that civilian leaders and the public will never know that the military is willfully challenging the principle of civilian control! Other types of behaviors, which are discussed more in chapter two of this dissertation, are far more obvious. For example, an Army general who writes an opinion piece strongly criticizing the President's foreign policy views challenges the principle of civilian control in that such a behavior likely undermines popular support for the President and imposes some sort of political cost that the President now has to contend with, particularly if the military officer who authored the piece is popular.

Can civilian leaders also violate the principle of civilian control? I contend that they can, but that when this occurs, it is typically the long term result of failing to ensure a climate of civilian control, rather than the result of a single act or behavior. This is consistent with recent scholarship by Beliakova (2021), who argues that one pathway through which the ``erosion'' of civilian control occurs is ``deference,'' that is, by civilians delegating too much power to the military \autocite{beliakova_erosion_2021}. Deference to the military resulting in harm could arise for a number of reasons. Perhaps civilian fail to assert themselves sufficiently during the course of a major military operation such as a war, or over the course of a lengthier time horizon. In the contemporary United States, for example, several critics have expressed concern that the principle of civilian control has been threatened for the past several years as a result of an extremely slow and politically charged confirmation process for senior civilian Department of Defense appointees, resulting in a shift in the overall balance of power within the Pentagon towards the uniformed military \autocite{seligman_civilian_2020}.

Finally, it should be pointed out that in the United States, the principle of civilian control is embedded within the US Constitution, which effectively establishes that the principle of civilian control involves the military obeying two civilian bosses. The first is the President, who is endowed as the Commander in Chief of the Armed Forces \autocite[Article 2, Section 2]{noauthor_us_1787}, and the second is Congress, to which the Constitution grants powers of oversight \autocite[Article 1, Section 8]{noauthor_us_1787}. Thus, from the US military's perspective, adhering to the principle of civilian control involves appeasing two different bosses - a prospect that is undoubtedly difficult at times.

\hypertarget{central-principle-2-non-partisanship-of-the-military-institution}{%
\section{Central Principle 2: Non-Partisanship of the Military Institution}\label{central-principle-2-non-partisanship-of-the-military-institution}}

The second central principle of democratic civil-military relations, particularly in the US context, is the principle of non-partisanship of the military institution. In general terms, the principle of non-partisanship of the military stipulates that the military not align itself - nor that it be made to align by civilians - with a particular political party or its platform. The principle of non-partisanship means in part that the military fully obeys the lawful orders of whichever political party is in office, yet it also means that the military cannot and will not identify itself, nor be made to identify by civilian leaders, as a partisan actor. Thus, both sets of actors, civilian and military, have a responsibility to maintain and adhere to the principle of non-partisanship of the military.

However, determining the degree to which civilian or military actors uphold the principle of non-partisanship in practice is a difficult undertaking, largely because it is difficult to both identify and measure what forms of behavior constitute a violation of this principle. After all, if militaries are, in fact, as Clausewitz and the critics of Huntington have contended, inherently political creatures who ``serve at the pleasure'' of their civilian bosses, then it stands to reason that the military will, at some point, enact the partisan policies, wishes, desires, and goals of their elected civilian leaders, who belong to a political party \autocite{mullen_chairman_2011}.

For this reason, scholars of civil-military relations have often separated the various meanings and implications of certain forms of behavior and their impact on politics. While there is no clear delineation of which forms of behavior and which contexts violate the principle of non-partisanship of the military, recent scholarship has attempted to create figurative space to allow greater acceptance - even encouragement - of military officers engaging in \emph{some} forms of non-partisan political behavior. For instance, scholars have argued that military officers and civilians should expect military leaders to increasingly contribute to debates about security policy on the grounds that doing so will lead to the better formulation of strategy and thus, to better national security outcomes \autocite{brooks_perils_2013,owens_military_2015,rapp_civil-military_2015}. In other words, scholars and senior military officers recognize that the military needs to be part of and embrace its role within the national security policy-making process.

Simultaneously, scholars and military officers alike recognize that other forms of behavior, such as retired military officers speaking at political conventions on behalf of political candidates, comprise blatant violations of the military's non-partisan ethic \autocite{dempsey_military_2016}. Thus, there is agreement that while there may be room for the military to engage in some political behaviors, there is a point at which other behaviors cross a figurative line such that they are not merely political, but primarily partisan in nature. It is at this juncture that such behavior becomes problematic.

Similar logic holds for civilian leaders with respect to the principle of non-partisanship of the military. In short, civilian leaders, according to the principle of non-partisanship, should not use the military in overtly partisan ways. Admittedly, there is far less clarity for civilian actors concerning when such usage of the military becomes inappropriate, precisely because of the fact that civilian leaders are elected to fulfill partisan campaign promises!

Several examples, nonetheless, are instructive. Many scholars and former military officers blasted the decision of President Trump to deploy US Troops to the US southern border in 2018, citing the move primarily as a political stunt ahead of the 2018 midterm elections \autocite{adams_trumps_2018}. Other critics alleged that President Trump's threat to use active duty forces to dispel protesters and rioters in the summer of 2020 likewise violated the military's non-partisan ethic \autocite{brooks_dismay_2020}. Finally, consider the case of President George W. Bush and his administration employing a number of senior retired military officers to boost popular support during a period of the Iraq War when both his policies and he personally were relatively unpopular. A fascinating yet somewhat disturbing picture emerges from Barstow (2008) as he describes efforts by the Bush Administration to court retired military officers for a major public opinion campaign. In particular, Barstow alleges that the Bush Administration politicized the latent popularity of dozens of retired military officials by first, warming up to them, and second, by prodding them to speak favorably to the media over several controversial issues, including the detention facility at Guantanamo Bay and the broader ``surge'' strategy to Iraq \autocite{barstow_behind_2008}.

The degree to which each of these examples constitute mainly civilian violations of the military's non-partisan ethic, or were simply indicative of politics as usual, is the subject of considerable civil-military relations debate.\footnote{In the next chapter, I emphasize that while some behaviors are clearly partisan, others are not as clear. For example, military officers speaking at political nominating conventions are clearly partisan acts. The behavior of retired military officers authoring op-eds, on the other hand, is not inherently partisan, but of course, this depends on the content of the op-ed.} What is clear, however, is that both sets of actors - military and civilian - have a responsibility to uphold the principle of non-partisanship, each by not aligning the military with overtly partisan goals and aspirations. However, where the line between appropriate political involvement and inappropriate partisan involvement of the military should be drawn is not always clear. This is especially true when examining the behavior of civilian leaders because they, by their very nature, are partisan entities. Still, the principle of non-partisanship, at least in general terms, applies: citizens should not be able to look at the military and to easily detect what their partisan preferences are.

\hypertarget{central-principle-3-the-non-interference-of-the-military}{%
\section{Central Principle 3: The ``Non-Interference'' of the Military}\label{central-principle-3-the-non-interference-of-the-military}}

The third central principle of American civil-military relations which I argue both Huntington and his critics affirm is what I am terming the \emph{``non-interference''} of the military into certain realms or areas of policy. This principle, I argue, is strongly implied in both Huntington's argument and critiques of his work. In short, the principle of non-interference simply means that there are policy arenas which are and should remain ``off limits'' for US military actors. By off limits, I mean that the military should not seek to influence some areas of government policy.

The origins of the principle of non-interference can be directly traced to Huntington's conception of military expertise. Recall that for Huntington, the concept of military expertise and military professionalism are inextricably linked, and involve the unique knowledge, skills, experiences, and behaviors of the military officer corps \autocite[7-18]{huntington_soldier_1957}. Huntington later devotes an entire chapter to the development of the notion of a unique and distinct ``military mind,'' which to him consists of the ``values, attitudes, and perspectives which inhere in the performance of the professional military function and which are deducible from the nature of that function'' \autocite[61]{huntington_soldier_1957}. In short, Huntington is arguing that the uniqueness and distinction of the military - in terms of its mindset, corporate ethic, and values - stems from the fact that the military's fundamental purpose is to defend the nation from external threats. The implication here is simple but profound: a professional military is one that is used only to fill roles that are directly related to national defense and/or the security of the nation.

And US history, certainly in the 20th Century at least, proves that this has indeed been the case. The US military has been primarily used to fight external rather than internal threats. Thus, there has been little if any need for the US military officer corps to assume any type of political role in managing the state's domestic problems. To be clear, the American military has certainly been used for a variety of non-traditional types of operations in the nation's history, and actually, quite often, from desegregating schools, responding to natural disasters, and most recently, assisting Federal authorities in responding to the Coronavirus Pandemic. But these types of missions, while frequent, have not come anywhere near to comprising the bulk of the US military's efforts.

Figure \ref{fig:nonint} is a graphical depiction of the placement of figurative civilian and military spheres. As I have noted above, Huntington largely imagined two distinct spheres, one political and one military, as shown on the left side of Figure \ref{fig:nonint}. His critics, on the other hand, instead argue that for both theoretical and practical reasons, the domain of the military is, while perhaps unique, nonetheless still part of the civilian leader's ultimate domain, as shown in the middle of Figure \ref{fig:nonint}.

Yet even these two different conceptions of civilian and military spheres have important similarities. Huntington and his critics would both also be against a military that is free to influence any purview or domain of civilian government that it wishes, as shown on the right side of Figure \ref{fig:nonint}. Even as those who strongly critique Huntington argue that the military is part of the political domain, nowhere do they argue that the inverse is true, i.e., that the entirety of the political domain is and should be open to the military. Said differently, the principle of non-interference affirms that conceptually, there is a rightful military domain that is necessarily limited. This is extremely important when it comes to the authority and influence of military leaders, and over what issues, problem sets, or topics these may be exercised.

The literature within comparative civil-military relations very much describes the principle of non-interference, even if in a descriptive way to denote that in many countries, such a principle does not exist! Indeed, militaries in many parts of the world influence and sometimes direct policy areas of the state to such an extent that doing so in the US context would be unthinkable. For instance, Stepan's concept of ``the new professionalism of internal security and national development'' traces the development of the Brazilian military's role expansion into domestic affairs as a result of having to primarily confront internal rather than external threats \autocite{stepan_new_1973}. But especially in democratic countries, a principle of ``non-interference'' of the military seems to be firmly established. This is true even as the exact boundaries of separation regarding which policy areas are off limits for military influence are not static over time nor across countries.\footnote{Even recently, civil-military relations scholars have started to much more earnestly explore how the tasks, missions, and roles of militaries change, and why this change matters with respect to the conduct of civil-military relations. In a special issue of \emph{The European Journal of International Security} released in February 2022, several authors explore how the concepts of military ``operational experiences'' and ``role conceptions'' - concepts that describe how military experiences (deployments in support of particular types of operations, for example) and the aspirational purposes for which militaries believe they primarily exist - shape civil-military relations. For example, see \textcite{harig_operational_2022} and \textcite{wilen_versatile_2022}.}

In his excellent work on Israeli civil-military relations, Yehuda Ben-Meir (1995) separates the affairs of the state into four broad areas, including political affairs, domestic affairs, national security, and the armed forces \autocite{ben-meir_civil-military_1995}. He argues that civilians should and do influence all four of these areas of politics, whereas the military should influence the three areas of domestic affairs, national security, and the armed forces \autocite{ben-meir_civil-military_1995}.\footnote{In terms of the activities which constitute political affairs, Ben-Meir includes items such as "taking control of the government (coups), influencing political appointments, or interfering in the decision making process. See \textcite{ben-meir_civil-military_1995}, 4-5 for an excellent description and diagram.} Ben-Meir is addressing the Israeli context, and thus, his paradigm may not map squarely onto the US context. This is not problematic, however. What is important for the purposes of my argument here is to highlight and to acknowledge, as Ben-Meir has, that the relative scope and specific areas of state policy that are influenced by military leaders is not the same as that of civilians.

Finally, it is important to briefly address what the principle of non-interference of the military implies for the conduct of the US military. The principle of non-interference implies that the military will not seek to perform, and that civilians will not assign the military to perform, roles or missions for which military forces are not suitably designed, nor those which are tangentially or indirectly related to national defense, except in cases of great crisis or need.

In a well known address given to the cadets at West Point in May of 1962, MacArthur captures well the spirit of the principle of ``non-interference'' by encouraging the soon-to-be officers to focus their careers on winning in combat, and leaving other issues for politicians to solve:

\SingleSpacing

\begin{quote}
\textbf{Yours is the profession of arms, the will to win, the sure knowledge that in war there is no substitute for victory, that if you lose, the Nation will be destroyed, that the very obsession of your public service must be Duty, Honor, Country. Others will debate the controversial issues, national and international, which divide men's minds. But serene, calm, aloof, you stand as the Nation's war guardians, as its lifeguards from the raging tides of international conflict, as its gladiators in the arena of battle. For a century and a half you have defended, guarded and protected its hallowed traditions of liberty and freedom, of right and justice. Let civilian voices argue the merits or demerits of our processes of government. Whether our strength is being sapped by deficit financing indulged in too long, by federal paternalism grown too mighty, by power groups grown too arrogant, by politics grown too corrupt, by crime grown too rampant, by morals grown too low, by taxes grown too high, by extremists grown too violent; whether our personal liberties are as firm and complete as they should be. These great national problems are not for your professional participation or military solution. Your guidepost stands out like a tenfold beacon in the night: Duty, Honor, Country} \autocite{macarthur_duty_1962}.
\end{quote}

\DoubleSpacing

A few other examples further illustrate the principle of non-interference. Hypothetically, most Americans, as well as the military, would probably not want a President to place the Pentagon in charge of designing a plan to overhaul social security, nor would they want the Chief of Naval Operations advocating that the Navy solve this problem for the nation. Non-hypothetical examples also abound. When the Chairman of the Joint Chiefs of Staff, General Mark Milley, foresaw no role for the military in the 2020 Presidential Election, he was implicitly invoking the principle of ``non-interference'' of the military, implying that it was not appropriate for the military to become involved in settling electoral disputes \autocite{silva_gen_2020}. Furthermore, across the world, the role of militaries in responding to the COVID-19 pandemic has raised numerous normative concerns over whether the military should occupy the roles that they have, including in the provision of healthcare and logistics, to include the distribution and contracting of vaccines.\footnote{At one point in late 2020, an active duty four star general apologized to the nation for a mix-up in information regarding the distribution of COVID vaccine, leading some critics to express concern that the military was making inherently political decisions that, from a normative perspective, posed some problems. See \textcite{passy_gen_2020} for more details.}

\hypertarget{summary-of-three-central-principles-of-us-civil-military-relations}{%
\section{Summary of Three Central Principles of US Civil-Military Relations}\label{summary-of-three-central-principles-of-us-civil-military-relations}}

In summary, Samuel Huntington and his critics both agree on the primacy and the importance of three central principles of civil-military relations. These principles are that of civilian control of the military, the non-partisanship of the military institution, and the ``non-interference'' of the military into certain arenas of state policy. When adhered to, followed, and respected, these principles generally constrain the behavior of both civilian and military actors in important ways, and help facilitate relatively harmonious civil-military relations.

Table \ref{tab:principles} captures these three principles, a concise definition of each principle, the actor that can violate each principle, and a few pertinent examples of behaviors that violate each principle.

\begin{table}

\caption{\label{tab:principles}Central Principles of US Civil-Military Relations}
\centering
\fontsize{10}{12}\selectfont
\begin{tabular}[t]{>{\raggedright\arraybackslash}p{6em}>{\raggedright\arraybackslash}p{12em}>{\raggedright\arraybackslash}p{12em}>{\raggedright\arraybackslash}p{12em}}
\toprule
Central Principle & Description & Military Example of Violation & Civilian Example of Violation\\
\midrule
\cellcolor{gray!6}{Civilian Control} & \cellcolor{gray!6}{Civilian Political Goals are Actualized and Implemented; Mechanisms of Civilian Oversight Function; No Overt Military Insubordination} & \cellcolor{gray!6}{Resigning in protest of policy; slow-rolling policy implementation; authoring an op-ed that criticizes a President's policy preferences} & \cellcolor{gray!6}{Failing to establish mechanisms and processes of oversight; delegating too much power to the military}\\
Non-Partisanship & The Military Institution Exists and Operates outside of Partisan Politics; Military Actors Fully Obey Political Leaders, and Do Not Advocate for Partisan Policies, People, or Platforms & Advocating for the platform of a political party or denouncing that of another; Declaring candidacy for partisan political office while in uniform & Urging several military generals and admirals to speak at a party political convention\\
\cellcolor{gray!6}{Non-Interference} & \cellcolor{gray!6}{There are Areas or Realms of State Policy making into which the Military does not Enter or Seek to Influence} & \cellcolor{gray!6}{Advocating that the President place the military in charge of overhauling social security} & \cellcolor{gray!6}{Appointing a serving unformed military officer as the Secretary of Labor or Education}\\
\bottomrule
\end{tabular}
\end{table}

\hypertarget{the-gap-in-the-scholarly-literature}{%
\chapter{The Gap in the Scholarly Literature}\label{the-gap-in-the-scholarly-literature}}

Having established the the baseline principles of civil-military relations as that of civilian control, non-partisanship, and non-interference, as well as the central role these principles play with respect to the facilitation of healthy democratic civil-military relations, I will briefly trace arguments made by scholars who seek to explain the military's role in politics. The goal of doing so is two-fold: first, to orient the reader to the value of this scholarship, and second, to explain the gap that currently exists within this scholarship.

Explaining why and how militaries intervene in politics is a massive topic within civil-military relations \autocites[for a sampling of this scholarship, see][]{finer_man_1962,taylor_politics_2003,croissant_beyond_2010,bove_beyond_2020,beliakova_erosion_2021}. Some works focus on specific types or forms of military intervention in politics, such as coups \autocite{horowitz_coup_1980,de_bruin_will_2019}, while others focus on explaining a range of intervention outcomes that can occur within a particular country or region of the world \autocites[for instance, see][]{stepan_new_1973,fitch_armed_1998}. I first summarize the contributions of six particular scholars whose works, in my view, are highly relevant to the topic this dissertation addresses. These are Samuel Finer's \emph{The Man on Horseback} (1957); Donald Horowitz's \emph{Coup Theories and Officer Motives: Sri Lanka in Comparative Perspective} (1980); J. Samuel Fitch's \emph{The Armed Forces and Democracy in Latin America} (1989); Bove, Rivera, and Ruffa's article,``Beyond Coups: Terrorism and Military Involvement in Politics'' (2020); Jeremy Teigen's \emph{Why Veterans Run} (2018); and Brian Taylor's \emph{Politics and the Russian Army} (2003). Then, I briefly describe several works that address political activities in mature democracies.

Finer (1962) is among the first of scholars to provide a comprehensive theory for why militaries intervene in politics. His central argument is that militaries intervene in politics as a broad result of having the ``motives and opportunities'' to do so \autocite[20-76]{finer_man_1962}, a framework that is adopted in this dissertation and further explained in the next chapter. Finer delineates several classes of motives, arguing that these motives may include a desire for the military to protect or enhance its own bureaucratic interests, as well as the belief or worldview that the military should serve as the ``custodian'' of the ``national interest,'' particularly in times of national crisis \autocite[30-31]{finer_man_1962}. Finer is also among the first of scholars to point out, in what is an insightful critique of Huntington that others have advanced in similar ways \autocite[for instance, see][]{brooks_paradoxes_2020}, that a strictly Huntingtonian concept of military professionalism might actually compel, rather than prevent, the military from intervening in politics \autocite[22-23]{finer_man_1962}.\footnote{Finer and Brooks argue that the concept of professionalism spurs military intervention into politics through different mechanisms. Finer argues that the desire to remain professional may lead military officers to refuse to obey a particular civilian leader because he or she can claim that in doing so, he or she is actually serving the broader concept of the state's interest, rather than obeying a particular civilian regime. In the US context, however, this is indeed the case, as the US military officer's oath is to support and defend \emph{the Constitution of the United States}, rather than to support or advance the agenda of a particular President. On the other hand, Brooks seems to warn about military officers who may, because of the belief that they are professionals, adopt a mindset such that the officer's actions will automatically carry little or any political ramifications. Both scholars are emphasizing that the concept of professionalism can be abused by military officers in such a way that elicits the very conduct that Huntington seeks to circumscribe.} Finer's major contribution, in my view, is theoretical, as he is the first to posit the significance of an underlying ``motives and opportunities'' framework to explain military intervention in politics. The main shortcoming of Finer's work, however, is that while it is theoretically rich, his dependent variable of military intervention is somewhat imprecise. Indeed, the dependent variables Finer uses are the ``levels of intervention'' undertaken by the military, which range from asserting ``influence'' to engaging in ``supplantment'' \autocite[77-116]{finer_man_1962}. These outcomes, and the range between them, include a significant amount of variation, however. While this range of outcomes is insightful, it does not facilitate granularity in examining the dependent variable.

Horowitz (1980) advances an understanding of Finer's ``motives and opportunities'' framework by examining the factors that produced an attempted military coup in Sri Lanka in 1962. Using qualitative interview data from nearly two dozen military officers who participated in the coup attempt, Horowitz analyzes the motives of the participating officers. He concludes, convincingly in my view, that identifying a specific motive, in the specific instance examined but also likely in any other attempted coup, is a complex undertaking, and that multiple different officer motives, to include organizational interests, societal cleavages, or bureaucratic politics, likely exist simultaneously \autocite[3-30, 179-221]{horowitz_coup_1980}. His main contribution is theoretical and empirical in that he highlights the dynamic and complex nature of officer motives, and in a way, warns against adopting a uniform explanation for the occurrence of coups. The main drawback to Horowitz's work again involves the dependent variable, as he is focused squarely on the explanation of coups.

A third scholar whose work touches on this dissertation is Fitch. Fitch seeks to explain how and why officers adopt unique beliefs regarding their normative roles in politics and within the government. Using rich interview data from the officers of several Latin American militaries in the 1980s, Fitch's develops the dependent variable of ``military role beliefs,'' and argues that officers adopt a particular belief as the result of their military's organizational culture and lessons learned from previous political interventions by the military \autocite[61-105]{fitch_armed_1998}. Fitch's dependent variable is different from that explored by either Finer or Horowitz, mainly in that it emphasizes the linkages between the beliefs adopted by an officer, and the general behaviors these officers engage in \autocite[see especially][65-100]{fitch_armed_1998}. This is an important contribution, in my view, because it explains how and why officers arrive at conclusions regarding how they should behave relative to civilians within government, and with respect to the principle of civilian control more broadly.

A fourth scholarly work is Bove, Rivera, and Ruffa's (2020) article, ``Beyond Coups: Terrorism and Military Involvement in Politics''. This work is important because it is among the first that explores the role of each type of actor, civilian and military, in spurring military involvement in politics. Using instances of terrorism as a theoretical starting point, the authors assert that military actors, by virtue of the information advantages they hold over civilians, ``push'' themselves into politics out of a desire to secure the state, while civilian leaders, who demand that the military employ its expertise by responding to terrorism quickly and successfully, ``pull'' the military into politics \autocite[268]{bove_beyond_2020}. However, the dependent variable used in this study is a composite measure of the military's involvement in politics based on the International Country Risk Guide (IRCG). This measurement is fine for the intended use that the authors have in mind, but it is not a measurement of specific forms or types of political behaviors that occur.

The fifth scholarly book that is highly related to my dissertation is Jeremy Teigen's (2018) \emph{Why Veteran Run: Military Service in American Presidential Elections, 1789-2016}. In the book, Teigen develops a comprehensive taxonomy to explain variation in the high percentage of final Presidential candidates with military service in different eras of US history. He points to a variety of different variables to explain this variation, including the type of military service of the candidate and the changing relationship between the state and its armed forces in US history \autocite{teigen_why_2018}. Teigen's work is all-encompassing and rich, and the puzzle he seeks to describe is not only relevant, but truly interesting. And while the political behavior that he examines is important, the book's focus is on one type of political behavior (running for the office of President).

Finally, Taylor (2003) provides another important contribution by applying Finer's motives and opportunities framework to the case of Russia \autocite[2]{taylor_politics_2003}. His impressive book, \emph{Politics and the Russian Army}, explores 19 incidents in which the Russian military intervenes (or might have intervened) in state ``sovereign power'' issues \autocite[3]{taylor_politics_2003}. Taylor's contribution is important because his application of a motives and opportunities framework to a particular context is precise and sound. Moreover, his analysis covers an impressive period of time - more than three centuries!

While the works of these scholars are incredibly rich and valuable, there are two broad shortcomings that this dissertation seeks to address. The first is an issue related to the dependent variable examined in each study. In several of the studies described above, the dependent variables and concepts of military intervention (in the case of Fitch, role beliefs, which is different yet similar) are generally wider and more extreme than the instances of political behavior that occur within mature democracies. In Teigen's book, which does examine a political behavior that occurs in the United States, the main shortcoming is that he mainly examines candidates who earn their respective party's nomination. He has to do this in order to explain the entirety of US history in his work, which he does, but this necessarily means that there are finer tuned aspects of particular campaigns that he cannot explore further in depth. In short, I can (and do) seek to employ many of the theoretical start points, including a general motives and opportunities framework, used by the authors that I have described above, but in a way that seeks to explain meaningful variation in specific forms of political behaviors that occur within mature democracies.

The second issue with the works described above, in my view, is that they tend to overlook the important role of civilian leaders in eliciting military intervention in politics. This is not the case for Bove, Rivera, and Ruffa (2020), as they accurately identify, in my view, the notion that civilian leaders seek to ``pull'' the military into politics at the same time that military leaders ``push'' themselves to do the same \autocite{bove_beyond_2020}. The other works described above, however, tend to focus primarily if not exclusively on the behavior or an outcome focused on military actors. This dissertation follows Bove and Ruffa (2020) in examining the simultaneous interplay that occurs between both sets of actors, civilian and military, but builds on their analysis by examining the role of domestic variables in explaining military intervention in politics rather than a type of threat (terrorism).

\hypertarget{how-this-dissertation-fills-this-scholarly-gap}{%
\section{How This Dissertation Fills This Scholarly Gap}\label{how-this-dissertation-fills-this-scholarly-gap}}

Thus, there is a gap in the civil-military relations literature regarding the causes and measurements of specific political activities involving the military within mature political democracies. To be clear, a number of other scholars have raised alarms about the impacts of rising political polarization on the conduct of US civil-military relations \autocites[for example, see][]{robinson_michael_danger_2018,burbach_partisan_2019,feaver_military_2020,feaver_we_2016,barno_how_2016,golby_jim_americas_2020,reid_retired_2020}, and others have examined several political behaviors by the US military over time, to include the endorsement of political candidates by retired military officers and social media habits by members of the military \autocite{griffiths_not_2019,dempsey_our_2010,urben_party_2013,urben_wearing_2014}. However, these works, while incredibly valuable, have nonetheless not attempted to develop an encompassing theoretical justification, nor a detailed measurement scheme, to explain variation in particular behaviors over time.

This dissertation differs from previous works and advances civil-military relations scholarship in several respects. First, this work seeks to develop a theory regarding the causes as well as a measurement strategy of political behaviors involving the military in democracies. In doing so, I conceptualize the three central principles of civil-military relations, which were described earlier in this chapter, as a necessary point of departure on the grounds that political behaviors are problematic insofar as the degree to which they violate one or more of these central principles. Second, this work harmonizes considerations that involve both types of actors, civilian and military, who can politicize the military. A perspective that considers simultaneous military and civilian efforts to politically engage with the military in various ways is consistent with previous research \autocite{bove_beyond_2020}, yet it also builds on recent efforts which have focused on one type of actor \autocites[for recent scholarship describing civilian politicization of the military, see][]{karlin_case_2020,golby_uncivil-military_2021}.

The main argument of this dissertation is that the confluence of two factors, political polarization and military prestige, greatly impacts the willingness of civilian and military actors to adhere to the central principles of civil-military relations in important ways. Furthermore, when civilian and military actors are less constrained to adhere to the principles of civil-military relations, each actor will behave in ways that violate these principles.

If this argument is true, there are several immediate repercussions that are worth briefly touching upon. The first is that to a significant extent, the political behaviors that occur involving the military are the product not primarily of mere circumstances and personalities, but rather circumstances and personalities that operate within a context marked by the broad political factors of polarization and military prestige. A second and related repercussion is that those who practice and observe civil-military relations might better understand the occurrence of, even if they do not (and I believe that they should not!) excuse the practice of, political behaviors that involve the military. An Army Chief of Staff or a Marine Corps Commandant have a difficult and demanding job during any point in time, but could it be more difficult when polarization is high, at least in terms of managing the behavior of those who serve in their respective organizations? Similarly, a President or a Secretary of State's duties and responsibilities are always difficult, but how are they made different and perhaps more difficult when the military is exceedingly prestigious? Unpacking, exploring, and understanding these dynamics is the goal of this dissertation.

\hypertarget{methodology-and-overall-plan-of-the-dissertation}{%
\chapter{Methodology and Overall Plan of the Dissertation}\label{methodology-and-overall-plan-of-the-dissertation}}

In chapter two, I develop a theory regarding the causes of political behavior involving the military in democracies. The theory posits that two variables, the levels of political polarization and military prestige, shape the degree to which military and civilian actors are constrained by the central principles of civil-military relations. The theory also argues that the three central principles of civil-military relations, which have been discussed in depth in this initial chapter, provide a solid foundation through which to look when identifying and measuring problematic political behaviors that involve professional militaries in democracies, particularly in the post-World War two era.

\hypertarget{chapter-3---retired-military-officer-opinion-commentary}{%
\section{Chapter 3 - Retired Military Officer Opinion Commentary}\label{chapter-3---retired-military-officer-opinion-commentary}}

Chapters three and four are quantitative, large-N studies of specific types of political behaviors undertaken by military (chapter 3) and civilian (chapter 4) actors. Chapter three investigates retired military officer opinion commentary by analyzing opinion commentary authored by retired US military officers over the past roughly four decades (1979-2020). This original analysis reveals that retired military officers are criticizing civilian officials at a greater frequency, adopting expressly partisan positions, and weighing in on topics that fall outside of traditional military expertise more frequently than in past years, and argues that these results are largely driven by increases in the level of political polarization.

\hypertarget{chapter-4---the-civilian-use-of-the-military-in-presidential-campaign-advertisements}{%
\section{Chapter 4 - The Civilian Use of the Military in Presidential Campaign Advertisements}\label{chapter-4---the-civilian-use-of-the-military-in-presidential-campaign-advertisements}}

Chapter four then examines a political behavior conducted by civilian actors - that of the content of Presidential campaign advertisements. Using data assembled from the Wisconsin Advertising Project and the Wesleyan Media Project, this chapter analyzes all television campaign advertisements aired during the five Presidential Elections held from 2000 - 2016, inclusively. This original analysis explores the degree to which military symbols and images appear in the advertisements, as well as the frequency with which various military figures appear in advertisements and engage in explicitly partisan behaviors, such as endorsing or attack political candidates. The statistical analysis reveals that the level of military prestige moderately drives changes in the way that civilian candidates harness military figures to engage in behaviors during campaign advertisements that challenge the principles of civil-military relations.

Chapters three and four are important because each chapter empirically demonstrates a link between a type of actor (military or civilian) and a particular form of political behavior (writing an op-ed or featuring a campaign advertisement in which a military figure engages in an explicitly partisan act). From a methodological standpoint, the quantitative analysis performed in each of these chapters assists in the disentangling of the variables of polarization and military prestige, which often vary in the same direction.

\hypertarget{qualitative-case-studies}{%
\section{Qualitative Case Studies}\label{qualitative-case-studies}}

After these large-N quantitative chapters, the dissertation then explores two case studies using the qualitative method of process tracing. These case studies are designed to examine not one instance of a political behavior, but rather the type and severity of multiple political behaviors that occur in a given era.

A case study, in the words of Brady and Collier (2010), ``may be understood as a temporally and spatially bounded instance of a specified phenomenon'' \autocite[208]{brady_rethinking_2010}. These same authors define process-tracing as involving ``the examination of `diagnostic' pieces of evidence within a case that contribute to supporting or overturning alternative explanatory hypotheses'' \autocite[208]{brady_rethinking_2010}. Brady and Collier also argue that the use of process tracing within a case study is valuable for at least two main reasons.

The first advantage of process tracing is that it can inform whether the independent variables influence the dependent variable through the pathways, mechanisms, and sequences claimed to be at work in a given theory \autocite[208-209]{brady_rethinking_2010}. In the context of this particular dissertation, this means that the use of case studies can help clarify the nature and the extent to which the level of political polarization serves as a ``motive'' for military actors to violate the central principles of civil-military relations, and the level of military prestige provides an ``opportunity'' for both civilian and military actors to do the same. By extension, this means that case studies can also help identify whether endogeneity is a problem within a given theory. The case studies should clearly illustrate that military and civilian actors alter the political behaviors they engage in that involve the military \emph{as a result} of changing levels of military's prestige or polarization, rather than the other way around.\footnote{In the next chapter, I argue that in United States context in particular, the most likely issue with endogeneity involves the relationship between military prestige and political behaviors. In other words, it is likely that the US military is deemed trustworthy by society in part because it does not engage in political behaviors that routinely violate principles such as civilian control or non-partisanship, or to say it differently, perhaps the level of military prestige declines if and when the military engages in acts that violate the principle of civilian control or non-partisanship. However, as I discussed in chapter two, endogeneity is less of a concern with respect to polarization, at least in the US context. This is mainly because I have defined polarization to be the contestation of worldviews, values, and moral judgments, and how these battle for primacy in the political sphere. These are likely formed prior to, rather than as the result of, the military's behavior.}

A second advantage of process tracing is that its use enables the evaluation of alternate explanations that may also be thought to influence the dependent variable \autocite[208]{brady_rethinking_2010}. In this dissertation, this means that process tracing can help inform the degree to which including changes in either norms or the prevailing threat environment, rather than changes in the level of polarization or military prestige, impact the ways in which civilian and military actors engage in behaviors that violate the principles of civil-military relations.

Ideally, both case studies in this dissertation would include one variable clearly changing while the other remains relatively constant, or at least changes relatively less than the other. Such a pattern will enable clearer discernment of the relative influences of each independent variables on the dependent variable. These insights suggest that each case study should consist of a finite period of time such that the number of other changes occurring over the same period are minimized. In short, the case studies explored in this dissertation should examine a short and focused time period in which one independent variable primarily changes.

\hypertarget{chapter-5---civilian-and-military-leaders-after-the-us-civil-war-1865-1885}{%
\section{Chapter 5 - Civilian and Military Leaders after the US Civil War (1865-1885)}\label{chapter-5---civilian-and-military-leaders-after-the-us-civil-war-1865-1885}}

Chapter five analyzes the behavior of military and civilian actors between the end of the American Civil War and roughly fifteen years later. Over this time period, the level of political polarization in the United States remained relatively high, but the level of overall military prestige decreased dramatically, driven sharply by a decline in the centrality of the overall role that the military filled to the nation after the Civil War and before an outbreak of war with Spain near the turn of the century.

It is an important period of history to analyze. Methodologically, the central principles that I have developed in this chapter did not yet exist, and yet, as I claim in the case study, it is clear to see how the levels of polarization and military prestige clearly influence how civilian and military actors behave with respect to the military and its leaders. In particular, I compare the types and characteristics of the political behaviors undertaken by civilian and military actors in the period immediately leading to the impeachment of Andrew Johnson with those undertaken 12-15 years later, when the Army intervened in several domestic labor disputes and when Congress debated implementing numerous Army reforms. I conclude that a decline in the level of military prestige that occurred during the Reconstruction years ultimately resulted in civilian and military actors engaging in relatively less visible and less extreme political behaviors than had occurred in the beginning of this period.

\hypertarget{chapter-6--civilian-and-military-leaders-before-and-after-the-gulf-war-1986-1996}{%
\section{Chapter 6 -Civilian and Military Leaders Before and After the Gulf War (1986-1996)}\label{chapter-6--civilian-and-military-leaders-before-and-after-the-gulf-war-1986-1996}}

Chapter six, the second case study, employs what is analogous to a regression discontinuity design by exploring the political behavior of military and civilian actors during the five years immediately before and after the US war against Iraq in 1990-1991. The US victory against Iraq is often pointed to by scholars as a significant event that served as a turning point in the US military's post-Vietnam renewal \autocite{kitfield_prodigal_1997}. From the late 1980s to the early 1990s, political polarization rises only slightly, but the Gulf War victory can be seen as a major boost in the level of military prestige. This case study thus involves an examination of a period that changes from relatively low polarization and low military prestige to one marked by low polarization and high military prestige.

The cases are roughly 100 years apart from each other, which will require that we identify important historical differences between the case. Still, the variable of military prestige varies significantly in each case study, but in different directions (from high to low in the first case study, and from low to high in the second), while the level of polarization remains relatively constant but at different relative levels (high in the first case study, and low in the second). These attributes and characteristics will help isolate the role that the level of military prestige plays, under various levels of polarization, in impacting the political behaviors that military and civilian actors commit that violate the central principles of civil-military relations.

\hypertarget{chapter-7---implications-and-conclusion}{%
\section{Chapter 7 - Implications and Conclusion}\label{chapter-7---implications-and-conclusion}}

In the seventh and final chapter, the dissertation's main findings are summarized and critical implications are discussed. This chapter also contains recommendations for future research. The central conclusion reached in this dissertation is that the sustainment of the three central principles of civil-military relations exposited in this chapter is simply not possible in prolonged eras of high political polarization.

\pagebreak

\printbibliography

\end{document}
